\documentclass{article} % For LaTeX2e
\usepackage{nips14submit_e,times}
\usepackage{amsmath}
\usepackage{amsthm}
\usepackage{amssymb}
\usepackage{mathtools}
\usepackage{hyperref}
\usepackage{url}
\usepackage{algorithm}
\usepackage[noend]{algpseudocode}
%\documentstyle[nips14submit_09,times,art10]{article} % For LaTeX 2.09

\usepackage{graphicx}
\usepackage{caption}
\usepackage{subcaption}

\def\eQb#1\eQe{\begin{eqnarray*}#1\end{eqnarray*}}
\def\eQnb#1\eQne{\begin{eqnarray}#1\end{eqnarray}}
\providecommand{\e}[1]{\ensuremath{\times 10^{#1}}}
\providecommand{\pb}[0]{\pagebreak}
\DeclarePairedDelimiter\ceil{\lceil}{\rceil}
\DeclarePairedDelimiter\floor{\lfloor}{\rfloor}

\newcommand{\E}{\mathrm{E}}
\newcommand{\Var}{\mathrm{Var}}
\newcommand{\Cov}{\mathrm{Cov}}

\def\Qb#1\Qe{\begin{question}#1\end{question}}
\def\Sb#1\Se{\begin{solution}#1\end{solution}}

\newenvironment{claim}[1]{\par\noindent\underline{Claim:}\space#1}{}
\newtheoremstyle{quest}{\topsep}{\topsep}{}{}{\bfseries}{}{ }{\thmname{#1}\thmnote{ #3}.}
\theoremstyle{quest}
\newtheorem*{definition}{Definition}
\newtheorem*{theorem}{Theorem}
\newtheorem*{lemma}{Lemma}
\newtheorem*{question}{Question}
\newtheorem*{preposition}{Preposition}
\newtheorem*{exercise}{Exercise}
\newtheorem*{challengeproblem}{Challenge Problem}
\newtheorem*{solution}{Solution}
\newtheorem*{remark}{Remark}
\usepackage{verbatimbox}
\usepackage{listings}
\title{Random Graphs: \\
Final Exam}


\author{
Youngduck Choi \\
CIMS \\
New York University\\
\texttt{yc1104@nyu.edu} \\
}


% The \author macro works with any number of authors. There are two commands
% used to separate the names and addresses of multiple authors: \And and \AND.
%
% Using \And between authors leaves it to \LaTeX{} to determine where to break
% the lines. Using \AND forces a linebreak at that point. So, if \LaTeX{}
% puts 3 of 4 authors names on the first line, and the last on the second
% line, try using \AND instead of \And before the third author name.

\newcommand{\fix}{\marginpar{FIX}}
\newcommand{\new}{\marginpar{NEW}}

\nipsfinalcopy % Uncomment for camera-ready version

\begin{document}


\maketitle

\begin{abstract}
Every human activity, good or bad, except mathematics, must come to an end. \\
- Paul Erdos
\end{abstract}

\bigskip

\begin{question}[1]
\hfill
\begin{figure}[h!]
  \centering
    \includegraphics[width=0.7\textwidth]{pm-f-1.png}
\end{figure}
\end{question}
\begin{solution} \hfill \\

\end{solution}

\newpage

\begin{question}[2]
\hfill
\begin{figure}[h!]
  \centering
    \includegraphics[width=0.8\textwidth]{pm-f-2.png}
\end{figure}
\end{question}
\begin{solution}
\textbf{(a)}
Let $\{X_i\}_{i \in I}$ 
be the collection of all subsets of $\Omega$,
with size $3$, such that the sum of the subset is 
precisely $n$, indexed by $I$. Let $B_i$
be the event $X_i \subseteq R$. In particular, 
$A = \bigwedge_{i \in I} \overline{B_i}$.
We note that
\eQb
\epsilon &=& P(B_i) = p^3
=  (cn^{-\frac{2}{3}})^3
= c^3 n^{-2} = o(1),
\eQe 
and
\eQb
\mu &=& E[X] = \sum_{i \in I} P(B_i) = 
{n-1 \choose 2} c^3 n^{-2} \sim c^3,
\eQe
as a simple
case counting with $x = i$ for $i = 1,..,n-1$,
shows that there are $
{ n-1  \choose 2}$ solutions to 
$x + y + z = n$, such that $x,y,z \in \Omega$,
and $x,y,z$ are distinct.  
To apply Janson, it remains to show that 
$\triangle = o(1)$. From the given combinatorial structure, we have the following equivalence:
\eQb
i \sim j &\iff& i \neq j \> \text{ and }
\> X_i \cap X_j \neq \emptyset \\
&\iff& 
|X_i \cap X_j | = 1, \\ 
\eQe
as $|X_i \cap X_j | = 2$ implies that the triple
is identical. Now, fix a solution $(x^*,y^*,z^*)$.
The number of solutions that share exactly one coordinate is clearly bounded above by $3n$. 
Therefore, the number of ordered pair $(i,j)$, with $|X_i \cap X_j| = 1$ is $O(n^3)$, as there are
$\Theta(n^2)$ of them in total.  
Therefore, it follows that
\eQb
\triangle &=& \sum_{i \sim j } P(B_i \wedge B_j) = \sum_{|X_i \cap X_j| = 1}
P(B_i \wedge B_j) \\
&=& O(n^3p^5) = O(n^{3} c^5 n^{-\frac{10}{3}}) = O(c^5 n^{-\frac{1}{3}}) = o(1).  
\eQe  
As $\epsilon = o(1)$ and $\triangle = o(1)$, Janson gives
\eQb
P(A) \to e^{-c^3},
\eQe
as required.

\bigskip

\textbf{(b)} Choose $p = n^{-\frac{2}{3}}\ln(n)^{\frac{1}{3}}$
 with foresight. Substituting the new parametrization to the above
calculation gives 
\eQb
\epsilon &=& P(B_i) = p^3
=  (n^{-\frac{2}{3}}\ln(n)^{\frac{1}{3}})^3
= n^{-2}\ln(n) = o(1),
\eQe
and 
\eQb
\mu &=& E[X] = \sum_{i \in I} P(B_i) = 
{n-1 \choose 2} n^{-2}\ln(n) \sim \ln(n),
\eQe
with
\eQb
\triangle &=& O(n^{-\frac{1}{3}} \ln(n)) =  o(1).
\eQe
Hence, Janson gives 
\eQb
P(A) = n^{-1+o(1)},
\eQe
as required.

\hfill $\qed$

\end{solution}

\newpage

\begin{question}[3]
\hfill
\begin{figure}[h!]
  \centering
    \includegraphics[width=0.8\textwidth]{pm-f-3.png}
\end{figure}
\end{question}
\begin{solution}
\hfill $\qed$
\end{solution}
\newpage

\begin{question}[4]
\hfill
\begin{figure}[h!]
  \centering
    \includegraphics[width=0.8\textwidth]{pm-f-4.png}
\end{figure}
\end{question}
\begin{solution}
\end{solution}
\newpage
\begin{question}[5]
\hfill
\begin{figure}[h!]
  \centering
    \includegraphics[width=0.8\textwidth]{pm-f-5.png}
\end{figure}
\end{question}
\begin{solution}
\hfill $\qed$
\end{solution}
\newpage
\begin{question}[6]
\hfill
\begin{figure}[h!]
  \centering
    \includegraphics[width=0.8\textwidth]{pm-f-6.png}
\end{figure}
\end{question}
\begin{solution}
\hfill $\qed$
\end{solution}
\newpage
\begin{question}[7]
\hfill
\begin{figure}[h!]
  \centering
    \includegraphics[width=0.8\textwidth]{pm-f-7.png}
\end{figure}
\end{question}
\begin{solution}
Let $\{A_i\}_{i\in I}$ be the collection of 4-cycles of $G$, and let $\{ X_i\}_{i \in I}$
be the corresponding indicator random variables. In particular, $X = \sum_{i \in I} X_i$. Now, 
by linearity of expectation, it follows that
\eQb
E[X] &=& \sum_{i \in I} P(A_i) \sim n^2p^4.
\eQe
Therefore, with $p = \dfrac{1}{\sqrt{n}}$, it follows that 

\hfill $\qed$
\end{solution}
\newpage
\begin{question}[8]
\hfill
\begin{figure}[h!]
  \centering
    \includegraphics[width=0.8\textwidth]{pm-f-8.png}
\end{figure}
\end{question}
\begin{solution}
\textbf{(a)} Let $X_i$ be the indicator of triangles, so $X = \sum_{i \in I} X_i$. By linearity of expectation,
we obtain
\eQb
E[X] &=& E[\sum_{i \in I} X_i] = \sum_{i \in I} E[X_i] 
= { n \choose 3 }p^3 = { n \choose 3 } \left(\dfrac{c}{n} \right)^3,
\eQe
and asymptotically 
\eQb
E[X] \sim \dfrac{c^3}{6},
\eQe
as required.

\smallskip

\textbf{(b)} 
From the definition of variance, it follows that
\eQb
Var[X] &=& \sum_{S} Var[X_S] + \sum_{S \neq T} Cov[X_S, X_T].
\eQe
In particular, the variance formula for 
discrete random variable yields 
\eQb
\sum_{S} Var[X_S] &=& {n \choose 3} p^3(1-p^3). 
\eQe
Since $p = o(1)$, we have that $(1-p^3) = o(1)$. 
Combining the result gives
\eQb
Var[X_S] &=& p^3 (1-p^3) \sim p^3 \> \text{ and } \> Var[X_S] \sim E[X_S] \sim \dfrac{1}{6}c^3. 
\eQe 
Now, observe that covariance is $0$ for $S,T$ pair, where $|S \cap T| \neq 2$. Now, for $S,T$ pair,
where $|S \cap T| = 2$, we have, by definition of covariance, 
\eQb
Cov(X_S, X_T) &=& E[X_S X_T] - E[X_S]E[X_T] = p^5 - p^6.
\eQe
Since there are ${n \choose 3} 3 (n-3)$ choices (fix the first triangle, pick the one that will not
be shared, and choose the remaining one from the rest of the graph), we finally have
\eQb
\sum_{S \neq T} Cov(X_S, X_T) &=& {n \choose 3} 3 (n-3) (p^5 - p^6) = o(1).
\eQe
Therefore, we can conclude that $Var[X] \sim \dfrac{c^3}{6}$ as well. 
 
\smallskip

\textbf{(c)} Using Janson's inequality (to be precise
a corolloary of it (pg.129),
we have that as $n \to \infty$
\eQb
P[X=0] \to e^{\frac{c^3}{6}},
\eQe
as required.

\hfill $\qed$
\end{solution}
\newpage
\begin{question}[9]
\hfill
\begin{figure}[h!]
  \centering
    \includegraphics[width=0.8\textwidth]{pm-f-9.png}
\end{figure}
\end{question}
\begin{solution}
For notational convenience, we replace $X_k$ with $X$.
Let $\{X_i\}_{i \in I}$ be the set of 
indicators defined on each k-clique of a complete
graph with $n$ vertices, indexed by $I$. In
particular, we have $\sum_{i \in I} X_i = X$.
By 
linearity of expectation, it follows that
\eQb
E[X] &=& E[\sum_{i \in I} X_i] = \sum_{i \in I}
E[X_i] = {n \choose k}\left( \dfrac{1}{3} \right)^{
{k \choose 2}}.
\eQe
Since $k$ is fixed, as $n \to \infty$, we have
\eQb
E[X] &\sim& \dfrac{n^k}{k!}(\dfrac{1}{3})^{
{k \choose 2}}
\eQe 
\hfill $\qed$
\end{solution}
\newpage
\begin{question}[10]
\hfill
\begin{figure}[h!]
  \centering
    \includegraphics[width=0.8\textwidth]{pm-f-10.png}
\end{figure}
\end{question}
\begin{solution}
\textbf{(a)}
We denote the event that $X_i$ indicates as $A_i$.
With the given setup, by linearity of expectation, we obtain
\eQb
E[X] &=& \sum_{0 \leq i < m } E[X_i] = \sum_{0 \leq i < m} P(A_i) \\
&=& \sum_{0 \leq i < m} 2^{-1-n} = 2^{-1}. 
\eQe

\bigskip

\textbf{(b)}

\bigskip

\textbf{(c)}

\bigskip

\textbf{(d)} By Brun's Sieve (theorem 8.3.1 from PM), it follows that


\hfill $\qed$
\end{solution}
\newpage
\begin{question}[11]
\hfill
\begin{figure}[h!]
  \centering
    \includegraphics[width=0.8\textwidth]{pm-f-11.png}
\end{figure}
\end{question}
\begin{solution}
\textbf{(a)} For simplicity, consider the complete 
graph case in $G(n,p)$, where we have 
pre-exposed a square in the gradation without 
its diagonals being exposed just yet. Now, 
exposing a diagonal edge will increase the number
of the triangles at least by $2$, so the 
Lipschitz condition is not satisfied. 

\bigskip

\textbf{(b)}
Let $Y$ be the max size family of edge disjoint triangles. Trivially, $X = 0 \iff Y=0$. Suppose that 
an edge exposure introduced $k$ new triangles to the current state of the graph. Consider
the collection of families 
of all edge disjoint triangles at the given exposure.
Each family can only increase by one, as they can pick
at most $1$ triangle out of the $k$ new triangles. Therefore, each family of edge disjoint triangles
increase at most by $1$ and the maximum number thus increase at most by $1$. We have shown that this
graph theoretic property is Lipschitz under the edge exposure gradation. 
 
\bigskip

\textbf{(c)} From the proof of theorem $7.3.2$ from PM, which is a direct consequence of the current setup 
and the Azuma's inequality, we obtain
\eQb
P[X=0] &=& P[Y=0] \leq P[Y - E[Y] \leq -E[Y]] \\
&\leq& e^{-\frac{E[Y]^2}{2{n\choose 2}}} \leq e^{-\Theta(\frac{n^2}{k^8})} = e^{-\Theta{(n^2)}},
\eQe
as required.

\hfill $\qed$
\end{solution}
\newpage
\begin{question}[12]
\hfill
\begin{figure}[h!]
  \centering
    \includegraphics[width=0.8\textwidth]{pm-f-12.png}
\end{figure}
\end{question}
\begin{solution}
\textbf{(a)}

\bigskip

\textbf{(b)}
From LLL, we can show that
\eQb
R(k,k) > \dfrac{\sqrt{2}}{e}k2^{\frac{k}{2}}(1+o(1)),
\eQe
which is a factor of $2$ improvement from the
union bound argument. 

\bigskip

\textbf{(c)}

\hfill $\qed$
\end{solution}
\newpage
\begin{question}[13]
\hfill
\begin{figure}[h!]
  \centering
    \includegraphics[width=0.8\textwidth]{pm-f-13.png}
\end{figure}
\end{question}
\begin{solution}
\hfill $\qed$
\end{solution}
\newpage
\begin{question}[14]
\hfill
\begin{figure}[h!]
  \centering
    \includegraphics[width=0.8\textwidth]{pm-f-14.png}
\end{figure}
\end{question}
\begin{solution}
\hfill $\qed$
\end{solution}
\newpage


\end{document}

