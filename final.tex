\documentclass{article} % For LaTeX2e
\usepackage{nips14submit_e,times}
\usepackage{amsmath}
\usepackage{amsthm}
\usepackage{amssymb}
\usepackage{mathtools}
\usepackage{hyperref}
\usepackage{url}
\usepackage{algorithm}
\usepackage[noend]{algpseudocode}
%\documentstyle[nips14submit_09,times,art10]{article} % For LaTeX 2.09

\usepackage{graphicx}
\usepackage{caption}
\usepackage{subcaption}

\def\eQb#1\eQe{\begin{eqnarray*}#1\end{eqnarray*}}
\def\eQnb#1\eQne{\begin{eqnarray}#1\end{eqnarray}}
\providecommand{\e}[1]{\ensuremath{\times 10^{#1}}}
\providecommand{\pb}[0]{\pagebreak}
\DeclarePairedDelimiter\ceil{\lceil}{\rceil}
\DeclarePairedDelimiter\floor{\lfloor}{\rfloor}

\newcommand{\E}{\mathrm{E}}
\newcommand{\Var}{\mathrm{Var}}
\newcommand{\Cov}{\mathrm{Cov}}

\def\Qb#1\Qe{\begin{question}#1\end{question}}
\def\Sb#1\Se{\begin{solution}#1\end{solution}}

\newenvironment{claim}[1]{\par\noindent\underline{Claim:}\space#1}{}
\newtheoremstyle{quest}{\topsep}{\topsep}{}{}{\bfseries}{}{ }{\thmname{#1}\thmnote{ #3}.}
\theoremstyle{quest}
\newtheorem*{definition}{Definition}
\newtheorem*{theorem}{Theorem}
\newtheorem*{lemma}{Lemma}
\newtheorem*{question}{Question}
\newtheorem*{preposition}{Preposition}
\newtheorem*{exercise}{Exercise}
\newtheorem*{challengeproblem}{Challenge Problem}
\newtheorem*{solution}{Solution}
\newtheorem*{remark}{Remark}
\usepackage{verbatimbox}
\usepackage{listings}
\usepackage{mathrsfs}
\title{ProbLimI: \\
Pset I}


\author{
Youngduck Choi \\
CIMS \\
New York University\\
\texttt{yc1104@nyu.edu} \\
}


% The \author macro works with any number of authors. There are two commands
% used to separate the names and addresses of multiple authors: \And and \AND.
%
% Using \And between authors leaves it to \LaTeX{} to determine where to break
% the lines. Using \AND forces a linebreak at that point. So, if \LaTeX{}
% puts 3 of 4 authors names on the first line, and the last on the second
% line, try using \AND instead of \And before the third author name.

\newcommand{\fix}{\marginpar{FIX}}
\newcommand{\new}{\marginpar{NEW}}

\nipsfinalcopy % Uncomment for camera-ready version

\begin{document}


\maketitle

\begin{abstract}
This work contains solutions to the exercises of the problem set I.
\end{abstract}

\bigskip

\begin{question}[1]
\hfill
\begin{figure}[h!]
  \centering
    \includegraphics[width=0.7\textwidth]{problim-e1-p1.png}
\end{figure}
\end{question}
\begin{solution} \hfill \\

\textbf{(ii)} Define $A_0 = \emptyset$ and 
\eQb
\tilde{A}_k &=& A_k \setminus A_{k-1} \> \> (k \in \mathbb{N}). 
\eQe  
Then, by countable additivity, we have
\eQb
\mathbb{P}(\bigcup_{k} 
\eQe

\end{solution}

\newpage

\begin{question}[2]
\hfill
\begin{figure}[h!]
  \centering
    \includegraphics[width=0.7\textwidth]{problim-e1-p2.png}
\end{figure}
\end{question}
\begin{solution} \hfill \\
\textbf{(i)}
As $\emptyset$ and $\Omega$ are in $\mathscr{G}_{\alpha}$ for all $\alpha$,
by the $\sigma -$field property of each $\mathscr{G}_{\alpha}$, 
it follows that $\emptyset, \Omega \in \bigcap_{\alpha}
\mathscr{G}_{\alpha}$. Now, it suffices to show
that
\eQb
A \in \bigcap_{\alpha} \mathscr{G}_{\alpha} &\implies& 
A^c \in \bigcap_{\alpha} \mathscr{G}_{\alpha}, 
\\
\{ A_n\} \subset \bigcap_{\alpha} \mathscr{G}_{\alpha}  &\implies& 
\bigcap_n A_n \in \bigcap_{\alpha} \mathscr{G}_{\alpha}.
\eQe
If $A \in \bigcap_{\alpha} \mathscr{G}_{\alpha}$ then, $A \in \mathscr{G}_{\alpha}$
for all $\alpha$, and by the $\sigma -$field assumption on each $\mathscr{G}_{\alpha}$,
it follows that $A^c \in \mathscr{G}_{\alpha}$ for all $\alpha$, so $A^c \in
\bigcap_{\alpha} \mathscr{G}_{\alpha}$. \\ 

\smallskip

If $\{A_n \} \subset \bigcap_{\alpha} \mathscr{G}_{\alpha}$, then $\{ A_n \} \subset
\mathscr{G}_{\alpha}$ for all $\alpha$, and by the $\sigma -$ field assumption on
each $\mathscr{G}_{\alpha}$, it follows that $\bigcap_n A_n \in \mathscr{G}_{\alpha}$
for all $\alpha$, so $\bigcap_n A_n \in \bigcap_{\alpha} \mathscr{G}_{\alpha}$. 

\smallskip

Now, recall that $\sigma(\mathscr{F})$ is defined to be the smallest $\sigma$-field
containing $\mathscr{F}$. Consider the family of $\sigma$-field that contains 
$\mathscr{F}$, and denote it by $\{ \mathscr{G}_\alpha \}$.  
The above result shows that $\bigcap_{\alpha} 
\mathscr{G}_{\alpha}$ is a $\sigma$-field, and it is trivial that it contains
$\mathscr{F}$. Obviously, for any $\alpha$, $
\bigcap_{\alpha} \mathscr{G}_\alpha \subset \mathscr{G}_{\alpha}$, 
which tells us that any $\sigma$-algebra
containing $\mathscr{F}$ contains $\bigcap_{\alpha} 
\mathscr{G}_{\alpha}$, so it follows that
$\bigcap_{\alpha} \mathscr{G}_{\alpha}$ is the smallest $\sigma$-algebra containing
$\mathscr{F}$ and notationally we have 
\eQb
\sigma(\mathscr{F}) &=& \{ \mathscr{F} \subset \mathscr{G} : \mathscr{G} \> 
\text{ is a } \sigma-\text{field} \},
\eQe 
as required. \hfill $\qed$ 

\textbf{(ii)}   

\end{solution}

\newpage

\begin{question}[3]
\hfill
\begin{figure}[h!]
  \centering
    \includegraphics[width=0.7\textwidth]{problim-e1-p3.png}
\end{figure}
\end{question}
\begin{solution} \hfill \\

\end{solution}

\newpage 

\begin{question}[4]
\hfill
\begin{figure}[h!]
  \centering
    \includegraphics[width=0.7\textwidth]{problim-e1-p4.png}
\end{figure}
\end{question}
\begin{solution} \hfill \\
\textbf{(a)} Let $f$ be a simple function, e.g. 
\eQb
f &=& \sum_{i=1}^{n} a_i X_{E_i},
\eQe
where $a_i \in \mathbb{R}$, $E_i \in \mathscr{F}$ pairwise disjoint
for $1 \leq i \leq n$, and $\bigcup_{i=1}^{n} E_i = \Omega$.
For sake of completeness, we show that $f$ is $(\mathscr{F},\mathscr{B}_{\mathbb{R}})$
measurable. For any $a \in \mathbb{R}$, observe that $f^{-1}((-\infty,a])$ 
is a union of sub-collection (allowing the empty collection) of $\{ E_i \}$, so
it is in $\mathscr{F}$. Hence, any simple function is $(\mathscr{F},
\mathscr{B}_{\mathbb{R}})$ measurable.  

\bigskip

Fix $a \in \mathbb{R}$. 
As $f^{-1}(-\infty) = \emptyset$ and 
$f$ is $(\mathscr{F},\mathscr{B}_{\mathbb{R}})$ measurable, it follows that 
\eQb
f^{-1}([-\infty,a]) &=& f^{-1}((-\infty,a]) \in \mathscr{F}.
\eQe
So, $f$ is $(\mathscr{F}, \mathscr{B}_{\mathbb{[-\infty,\infty]}})$ measurable.

\bigskip

\textbf{(b)} Observe that 
\eQb
\liminf_{n \to \infty} X_n &=& \inf_k \sup_{n \geq k} X_n\\
\limsup_{n \to \infty} X_n &=& \sup_k \inf_{n \geq k} X_n\\.
\eQe
Hence, with symmetry of $\inf$ and $\sup$, it suffices to show that
$\sup_n X_n$ is measurable.


Fix $a \in \mathbb{R}$. Then, we have
\eQb
(\sup_n X_n)^{-1}([-\infty,a]) &=& \bigcap_n X_n^{-1} ([-\infty,a]) \in \mathscr{F}.
\eQe 
So, $\sup X_n$ is measurable. 

\bigskip

Let $\mathscr{G}$ be a class of functions such that $(a)$ and $(b)$ are true. 
We wish to show that $m\mathscr{F} \subset \mathscr{G}$. By $(a)$, we know that
simple functions are in $\mathscr{G}$. Now, if $f \in m\mathscr{F}$, then
by the simple approximation lemma, there exists a sequence of simple functions
$\{X_n \}$ such that $X_n$ converges pointwise to $f$. Then, by $(b)$,
\eQb
f = \limsup_{n \to \infty} X_n \in \mathscr{G},
\eQe 
so $m\mathscr{F} \subset \mathscr{G}$, and $m\mathscr{F}$ is the smallest class
of functions satisfying properties $(a)$ and $(b)$.

\end{solution}
\newpage

\end{document}
